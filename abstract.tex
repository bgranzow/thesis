\specialhead{ABSTRACT}

In a finite element simulation, not all of the computed
data is of equal importance. Rather, the goal of an engineering
practitioner is often to accurately assess only a small
number of critical outputs, such as the displacement
at a point or the von-Mises stress over a domain. When
these outputs can be expressed as functionals,
a strategy known as \emph{adjoint-based error estimation}
can be employed to accurately assess output errors.
Using this error information, mesh adaptation can then
be utilized to reduce and control output errors.
The use of adjoint-based error estimation and mesh adaptation
is much more prevalent in computational fluid dynamics
applications when compared to computational solid mechanics.
This can in part be explained by the high level of expertise
required to derive and implement adjoint-based error estimation
routines in computational solid mechanics.

In this thesis, we present an approach to automate
the process of adjoint-based error estimation and mesh
adaptation to lower the barrier of entry for solid
mechanics practitioners. This approach has been developed
to be applicable to both Galerkin and stabilized finite
element methods, but we mainly emphasize stabilized finite
elements. In particular, we demonstrate the
effectiveness of this approach for two and three
dimensional problems in incompressible elasticity and
elastoplasticity. Further, we demonstrate the ability
of this approach to execute effectively on parallel
machines.

The variational multiscale (VMS) method is a particular
methodology that allows one to develop a stabilized
finite element method. As a further research endeavor,
we develop and investigate a novel approach for
adjoint-based error estimation and mesh adaptation
for VMS methods. In particular, we develop an approach
for adjoint enrichment based on VMS techniques.
